
%%%----------------------------------------------------------     
\chapter{System Documentation}
%%%----------------------------------------------------------
%Give a well-structured description of the architecture and the technical design of your implementation, with sufficient granularity to enable an external person to continue working on the project.
\newcommand{\ComputeEnergy}{\texttt{compute\_energy}}
\newcommand{\Estimate}{\texttt{estimate}}
\newcommand{\Lbp}{\texttt{lbp}}
\newcommand{\Utils}{\texttt{utils}}
\newcommand{\Params}{\texttt{params}}

\section{Dependencies Installation}
%show how to install opencv and various requirements
The system makes use of several external libraries, which can be easily installed using \texttt{pip} and the \texttt{requirements.txt} file:
\begin{lstlisting}[stepnumber=0]
pip install -r requirements.txt
\end{lstlisting}
\begin{tcolorbox}[title=Note:]
\textbf{Python 2.7} is required in order to run the system, consequently, the appropriate version of \textbf{pip} should be used. 
\end{tcolorbox}

\section{Running the System}
%In order to estimate the depth-map of a 
The system can be executed by invoking the following instruction from the operating system's command line terminal:
\begin{lstlisting}[stepnumber=0]
	python estimate.py [-i][-b] <config.txt>
\end{lstlisting}
The argument \texttt{<config.txt>} is a configuration file containing parameter values and paths to the folder storing all data necessary for the disparity-maps recovering process (this data is generally either pictures or other disparity-maps).
The options \texttt{-i} and \texttt{-b} are used to communicate the system which  of the available phase should be executed:
\begin{description}
	\item[\texttt{-i}] it executes only Disparity Initialization; the resulting disparity-maps are then stored in an output folder.
	\item[\texttt{-b}] it executes only Bundle Optimization. Note that ,because of how bundle optimization works, it also requires as input  the previously estimated depth-maps. As before, the resulting disparity-maps are then stored in an output folder.
	\item[\texttt{-ib}] it executes both Disparity Initialization and Bundle Optimization.
\end{description}

\section{System's Input}
\newcommand{\ConfigParam}[1]{\texttt{"#1"}}

\subsection{Camera File}
This file contains information about the camera parameters; in particular, it stores information 
about intrinsic matrix, rotations, and translation of the camera for each picture in the sequence.
The camera file itself is quite simple; for each image the intrinsic matrix, 
rotation matrix, and position must be included in plain text, one after the other, without interleaving blank lines.

The code below shows an example of such format: the first three rows are the \emph{intrinsic matrix},
the second three rows are the \emph{rotation matrix}, and the final row is the \emph{position vector}.
\begin{lstlisting}[language=C]
1139.7929     0.0000000     479.50000
0.0000000     1139.7929     269.50000
0.0000000     0.0000000     1.0000000
1.0000000     0.0000000     0.0000000
0.0000000     1.0000000     0.0000000
0.0000000     0.0000000     1.0000000
0.0000000     0.0000000     0.0000000
\end{lstlisting}
\dotfill

This pattern must be repeated for each image in the target sequence, and two blank lines must be used between data of different images.
Also, the first line of the camera file must be a number, indicating the number of camera frames in the file itself. A blank line must follow. A complete example of a camera file which makes use of a sequence with three frames is the following:
\begin{lstlisting}[language=C]
3

1139.7929     0.0000000     479.50000
0.0000000     1139.7929     269.50000
0.0000000     0.0000000     1.0000000
1.0000000     0.0000000     0.0000000
0.0000000     1.0000000     0.0000000
0.0000000     0.0000000     1.0000000
0.0000000     0.0000000     0.0000000


1139.7929     0.0000000     479.50000
0.0000000     1139.7929     269.50000
0.0000000     0.0000000     1.0000000
0.9999972     -0.001567     0.0017456
0.0015681     0.9999985     -0.000634
-0.001744     0.0006373     0.9999982
-0.320146     0.0141622     -0.049300


1139.7929     0.0000000     479.50000
0.0000000     1139.7929     269.50000
0.0000000     0.0000000     1.0000000
0.9999884     -0.002738     0.0039581
0.0027459     0.9999945     -0.001806
-0.003953     0.0018177     0.9999905
-0.720470     -0.028543     -0.089562
\end{lstlisting}
\dotfill
\begin{Note}
The location of the camera file is defined by the configuration parameter 
\ConfigParam{camera\_file}.
\end{Note}


\subsection{Pictures Directory}
Directory which contains the picture sequence to be used during the estimation process.
The images must be named using the syntax \texttt{img\_<num>}, with \texttt{<num>} numeric string with value in the range between \texttt{0000} ang \texttt{9999}. The images filenames must be organized in a 
contiguous fashion: no gaps should be present between the smallest and largest filename numbers.
All such image files should have same format and resolution.\\
\begin{Note}
The image format, as well as the location of the directory are indicated by the configuration parameters
\ConfigParam{pictures\_file\_extension} and\ConfigParam{pictures\_directory} respectively. 
\end{Note}

\subsection{Depth-maps Directory}\label{sec:depthmaps_dir}
Directory containing a sequence of disparity-maps. These can then be used during execution of \cref{alg:energy_data} in order to perform the Bundle Optimization. It should be noted that these disparity-maps are generally the result of a previous invocation of the system.

The disparity-maps are stored as \textbf{.npy} files.
Disparity-maps stored inside this directory shoulb be named  \texttt{depth\_<num>}, with  \texttt{<num>} numeric string which assumes values inside the range \texttt{0000} to \texttt{9999}. As with the images, the depth-maps filenames must be organized in a contiguous fashion: no gaps should be present between the smallest and largest filename numbers.\\
\begin{Note}
The location of the directory is indicated by the configuration parameter \\ \ConfigParam{depthmaps\_directory}.
\end{Note}

\section{System's Output}
The output of the system is a disparity-map directory, as such, names and format follows the same convention as described in \cref{sec:depthmaps_dir}. The location of the output directory is indicated by the configuration parameter \ConfigParam{output\_directory}.

%Different folders and files are used by the system during depth estimation; these can be divided into three types: \emph{configuration file}, \emph{picture-folder}, and finally \emph{depth-folder}.

%\paragraph{Configuration File.} The configuration file is used to set the parameter values to be used while running the system, it must contain a JSON object with key-value pairs.

%These pairs specify the parameter values used in the various algorithms, as well as more practical information, like the directories storing the input image sequence or the output depth-maps. Generally speaking, this file contains all the information necessary for the system in order to work properly.
%Most of these parameters have default values,so they are not required to be present in the configuration file, that is, with the exception of the first few lines: indeed, the first few lines in the file must set the values for the keys: \texttt{picture\_folder}, containing the input pictures, and \texttt{depth\_folder\_output}, which will store the estimated depth-maps.

%\paragraph{Picture Folder.} This folder contains the picture sequence to be used during the estimation process. The images must be named using the syntax \texttt{img\_\textit{<num>}}, with \texttt{\textit{<num>}} numeric string which assumes values from \texttt{0000} to \texttt{9999}. The images filenames must be organized in a contiguous fashion: no gaps should be present between the smallest and largest filename numbers. All the image files should have format \texttt{.png} or \texttt{.jpg}, and same resolution.


%\paragraph{Depth Folder.} This folder contains the output depth-maps, stored as \texttt{.npy} files. This is standard binary format used by \textbf{NumPy} to store a single \textbf{NumPy} array. These files are named \texttt{depth\_\textit{<num>}}, as before,  \texttt{\textit{<num>}} can assume values between \texttt{0000} and \texttt{9999}; said number indicates the image used to extract the depth-map.

%\paragraph{Output Folder}
\section{Configuration File}
File containing information used by the system in order to estimate and store depth-maps from a sequence of images. The file must contain a single \JSON{} objects; the fields of said objects are then used by the system during execution. An example of a configuration file is as follows:
\begin{lstlisting}[language=C,basicstyle=\ttfamily, numbers=left]
{
	"camera_file" : "camera.txt",
	"depthmaps_directory" : "depth",
	"pictures_directory" : "source",
	"output_directory" : "output",
	"pictures_file_extension" : ".png",
	"height" : 270,
	"width" : 480,
	"start_frame" : 9,
	"end_frame" : 30
	"depth_levels" : 100,
	"depth_min" : 0.0,
	"depth_max" : 0.008,
	"epsilon" : 50.0,
	"sigma_c" : 10.0
}
 
\end{lstlisting}
\dotfill

\subsection{Parameters}
A number of parameters are required to be in the configuration file if the correct behavior of the system is desired. A descriptive list of all the required parameters can be found at \cref{tab:req_params}.
The most important between these are \ConfigParam{camera\_file}, \ConfigParam{pictures\_directory}, and \ConfigParam{output\_directory}. Indeed, they are used to manage the files and directories that are used by the system.

Other important parameters are \ConfigParam{depth\_min}, \ConfigParam{depth\_max}, and \ConfigParam{depth\_levels} which are used to respectively define the value of $d_{min}$, $d_{max}$, and the quantization precision.\\
\begin{Important}
	The optional parameter \ConfigParam{depthmaps\_directory} is required in order to execute Bundle Optimization during the estimation process. The parameter's value determines the directory containing the initialized disparities used by the geometry constraint.
\end{Important}

{
\newcommand{\String}{\textbf{String}}%
\newcommand{\Integer}{\textbf{Int}}%
\newcommand{\Float}{\textbf{Float}}%

\begin{table}[h]
\begin{tabular}{|p{.35\textwidth}|p{.45\textwidth} | p{.12\textwidth}|}
	\hline
	\textbf{Required Parameters}&\textbf{Description}&\textbf{Type}\\
	\hline
	\ConfigParam{camera\_file} 
	& File containing the the camera parameters.
	& \String\\\hline
	%
	\ConfigParam{pictures\_directory} 
	& Directory containing the images used during depth-maps recovery.
	&\String\\\hline
	%
	\ConfigParam{output\_directory} 
	& Directory in which the output disparity-maps are stored.
	&\String\\\hline
	%
	\ConfigParam{pictures\_file\_extension} 
	& File extension used by the images in the pictures directory.
	&\String\\\hline
	%
	\ConfigParam{height} 
	& Height of the output disparity-maps. It can be different from the height of the original images.
	&\Integer\\\hline
	%
	\ConfigParam{width} 
	& 	Width of the output disparity-maps. It can be different from the width of the original images.
	&\Integer\\\hline
	%
	\ConfigParam{start\_frame} 
	& Frame from which the disparity-maps recovery process will start. For example if it has value $10$, then the  first picture whose disparity-map is computed will be \texttt{img\_0010}.
	&\Integer\\\hline
	%
	\ConfigParam{end\_frame} 
	& Ending frame used for disparity-map recovery.
	&\Integer\\\hline
	%
	\ConfigParam{depth\_min} 
	& Minimum admissible disparity-value used during recovery.
	&\Float\\\hline
	%
	\ConfigParam{depth\_max} 
	& Maximum admissible disparity-value used during recovery.
	&\Float\\\hline
	%
	\ConfigParam{depth\_levels} 
	& Number of discrete disparity labels used during estimation.
	&\Integer\\\hline
	%
	\hline
\end{tabular}
\caption{Table indicating all the parameters necessary in order to run the system without errors.}\label{tab:req_params}
\end{table}
}
%"depthmaps\_directory" Directory containing the disparity-maps used during Bundle Optimization.
%"window\_side" Size of the window used when estimating photo/geometry consistency of pixels in \cref{sec:init_alg} and \cref{sec:bundle_alg}.A bigger window size means a greater number of samples.
%"sigma\_c" 	Parameter used in function %init
%"sigma\_d" 	Parameter used in function %bundle
%"eta" 		Parameter used in function %lbp
%"w\_s" 		Parameter used in function %lambda
%"epsilon" 	Parameter used in function %lambda



\section{Python Modules}
The implementation make use of a number of custom python modules:

{
\centering
\begin{tabular}{p{.18\textwidth} p{.7\textwidth}}
	\hline\hline
	\ComputeEnergy& 
	This module contains functions necessary for the computation of the data cost in \cref{eq:init_energy} and \cref{eq:init_energy_data}, it also contains the function that computes the smoothness weights $\lambda(\cdot,\cdot)$.\\\hline
	%
	\Estimate& 
	This module is used to estimate the disparity-maps and subsequently store them in the output directory. To do so, it makes use of the module \ComputeEnergy{} and \Lbp. This module can be seen as the main entry-point of the system.\\\hline
	%
	\Lbp&
	This module contains an implementation of the \emph{Loopy Belief Propagation} algorithm used by the \Estimate{} module during the depth-maps estimation process.\\\hline
	%
	\Utils&
	This module contains a variety of class used by the other custom python modules in order to pass data between function calls, as well as debugging and visualization procedures.\\
	\hline\hline
\end{tabular}
}\\

To see the actual procedures and classes used by these modules, the reader should refer to the project's public repository \ProjectUrl{} and documentation.   

%\end{table}





